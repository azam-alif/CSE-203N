\documentclass{article}
\usepackage{graphicx} 
\usepackage{listings}
\usepackage{xcolor}
\usepackage{geometry}
\geometry{top=2cm, bottom=2cm, left=2cm, right=2cm}

\title{CSE 203N}
\author{2021132040}
\date{Submission date: 08/04/2025}

\begin{document}

\maketitle

\lstset{
  language=C,
  basicstyle=\ttfamily\small,
  keywordstyle=\color{blue},
  commentstyle=\color{gray},
  stringstyle=\color{red},
  frame=single,
  columns=flexible,
  showstringspaces=false,
  showspaces=false
}

\section*{Problem-1} C program to check whether an integer is an Armstrong number.
\begin{lstlisting}
#include<stdio.h>

int armstrong(int num, int power);
int size(int num); //to get the length of the integer

int main(){
    int num, sum, power;
    scanf("%d",&num);
    power = size(num);
    sum = armstrong(num, power);
    if(sum == num){
        printf("%d is an Armstrong number.\n", num);
    } else {
        printf("%d is not an Armstrong number.\n", num);
    }
    return 0;
}

int armstrong(int num, int power){
    int sum=0, remainder,i;
    while(1){
        int temp = 1;
        remainder = num % 10;
        for(i=0; i<power; i++){
            temp = temp*remainder;
        }
        sum += temp;
        num = num / 10;
        if (num == 0){
            break;
        }
    }
    return sum;
}

int size(int num){
    int count = 0;
    while(1){
        num = num/10;
        count++;
        if (num == 0){
            break;
        }
    }
    return count;
}
\end{lstlisting}

\section*{Problem-2} C program to reverse an integer number.
\begin{lstlisting}
#include<stdio.h>
int main(){
    int a, reverse = 0, remainder;
    printf("Enter a number: ");
    scanf("%d", &a);
    do{
        remainder = a%10;
        reverse = reverse * 10 + remainder;
        a = a/10;
    } while (a != 0);
    printf("reverse = %d\n", reverse);
    return 0;
}
\end{lstlisting}
\section*{Problem-3} C program for printing the larger among the sums and odd numbers in an array. 
\begin{lstlisting}
#include<stdio.h>
int main(){
    int n, i, even_sum = 0, odd_sum = 0;
    printf("Enter the length of the array: ");
    scanf("%d",&n);
    int arr[n];
    printf("Enter values: ");
    for(i=0; i<n; i++){
        scanf("%d", &arr[i]);
        if(arr[i]%2 == 0){
            even_sum += arr[i];
        } else {
            odd_sum += arr[i];
        }
    }
    if(even_sum > odd_sum){
        printf("Sum of even numbers is greater. Which is %d\n", even_sum);
    } else if(even_sum == odd_sum) {
        printf("Sum of even numbers is equal to the sum of  odd numbers. Which is %d\n", even_sum);
    } else {
        printf("Sum of odd numbers is greater. Which is %d\n", odd_sum);
    }
    return 0;
}
\end{lstlisting}

\newpage
\section*{Problem-4} C program for counting the duplicate integer in an array. 
\begin{lstlisting}
#include <stdio.h>

int main() {
    int n, i, j, count = 0;
    printf("Enter length of the array: ");
    scanf("%d", &n);

    int arr[n];
    printf("Enter array elements:\n");
    for(i = 0; i < n; i++)
        scanf("%d", &arr[i]);

    for(i = 0; i < n; i++) {
        for(j = i + 1; j < n; j++) {
            if(arr[i] == arr[j]) {
                count++;
                break; // avoiding counting the same duplicate again
            }
        }
    }

    printf("Total number of duplicate elements: %d\n", count);
    return 0;
}
\end{lstlisting}
\end{document}

